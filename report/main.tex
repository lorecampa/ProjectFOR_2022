\documentclass{article}

\usepackage{amsmath} % for math formatting

\begin{document}

\title{Biogas Location (Big Lab Project)}
\author{Lorenzo Campana, Sandro D'Andrea}
\date{\today}
\maketitle

\section{Introduction}
This problem can be modeled as a mixed integer linear programming (MILP) problem, where we need to decide which 
farms to locate the plants at and which farms to assign to each plant as suppliers.

\section{Problem}
An association of $n$ farmers wants to open $p$ plants to produce energy from biogas.
Each plant will be opened at a farm of a member of the association and will be powered with corn chopping purchased from the farm itself or from other neighboring farms.
Each farm $i$ can provide at most $c_i$ tons of corn chopping, with a percentage of dry matter $a_i$. As you may know, dry matter is the key component of corn chopping used for biogas production. In order to maintain the quality of produced energy, each plant must burn a mixture of corn chopping with a percentage of dry matter between $k_{min}$ and $k_{max}$.
At most one plant can be located in each farm, and every farm can sell its corn chopping to one and only one plant.
Each farm $i$ is located at coordinates $x_i$ and $y_i$, representing respectively its latitude and longitude, and the cost of moving corn chopping from a farm $i$ to a farm $j$ is proportional to the euclidean distance between the two farms (it does not depend on the actual quantity moved, since the trucks used for this transportations are sufficiently big).
Under such conditions, every plant produces $Q$ kWh of energy per ton of corn chopping burned. The energy produced by each plant will be fed into the national electricity system, at a unitary price of $b$ (€/kWh). Moreover, due to state regulations, each plant must not produce more than $M$ kWh of energy.
You must locate $p$ plants among the available farms and assign the farms that will supply each plant, with the goal of maximizing the total revenues of the association.
Let's define problem set and parameters:
\begin{itemize}
    \item $I$ = set of farms
    \item $n$ = number of farms
    \item $p$ = number of plants to locate
    \item $b$ = revenue per unit of energy (€/kWh)
    \item $M$ = max energy production (kWh)
    \item $Q$ = energy produced by a ton of corn chopping (kWh/t)
    \item $k_{min} (k_{max})$ = min (max) percentage of dry matter for fermentation
    \item $a_i$ = percentage of dry matter in chopping from farm $i \in I$
    \item $c_i$ = tons of corn chopping available for each $i \in I$ (t)
    \item $x_i, y_i$ = coordinates of farm $i \in I$
\end{itemize}

Let's define the following variables:
\begin{itemize}
    \item $x_i$: binary variable that is 1 if a plant is located at farm $i$ and 0 otherwise.
    \item  $y_{i,j}$: binary variable that is 1 if farm $j$ supplies corn chopping to farm $i$ and 0 otherwise.
    \item  $z_{i,j}$: variable that represent the number of tons of corn choppings that farm $j$ supplies to farm $i$.

\end{itemize}

The objective function is to maximize the total revenues of the association, which is the sum of the revenues from each plant. The revenue from each plant is equal to the energy produced by the plant multiplied by the unitary price of energy. The energy produced by each plant is equal to the sum of the energy produced by each ton of corn chopping burned, multiplied by the number of tons of corn chopping burned.

Therefore, the objective function can be written as:

$$\text{maximize} \sum_{i=1}^n \sum_{j=1}^n (Q \cdot b \cdot z_{i,j} - y_{i,j} \cdot d_{i,j})$$

We need to add constraints to ensure that each farm can 
only be assigned to at most one plant, and that each farm can only supply corn chopping to at most one plant. 
We can add these constraints as follows:

$$\sum_{i=1}^n x_i = p$$

$$\sum_{i=1}^n y_{i,j} \leq 1 \quad \forall j$$

$$ z_{i,j} \leq y_{i,j} \cdot c_j \quad \forall i \quad \forall j$$

We also need to ensure that the percentage of dry matter of the corn chopping burned at each plant is within the required range. We can add this constraint as follows:

$$k_{min} \leq \frac{\sum_{j=1}^n z_{i,j} \cdot a_j}{\sum_{j=1}^n z_{i,j}} \leq k_{max} \quad \forall i$$

Finally, we need to ensure that each plant does not produce more than the maximum allowed amount of energy. We can add this constraint as follows:

$$\sum_{j=1}^n z_{i,j} \cdot Q \leq M \quad \forall i$$

\end{document}
